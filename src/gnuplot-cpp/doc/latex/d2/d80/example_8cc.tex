\section{example.cc File Reference}
\label{d2/d80/example_8cc}\index{example.cc@{example.cc}}
{\tt \#include $<$iostream$>$}\par
{\tt \#include \char`\"{}gnuplot\_\-i.hpp\char`\"{}}\par
\subsection*{Defines}
\begin{CompactItemize}
\item 
\#define {\bf SLEEP\_\-LGTH}~2
\item 
\#define {\bf NPOINTS}~50
\end{CompactItemize}
\subsection*{Functions}
\begin{CompactItemize}
\item 
void {\bf wait\_\-for\_\-key} ()
\item 
int {\bf main} (int argc, char $\ast$argv[$\,$])
\end{CompactItemize}


\subsection{Define Documentation}
\index{example.cc@{example.cc}!NPOINTS@{NPOINTS}}
\index{NPOINTS@{NPOINTS}!example.cc@{example.cc}}
\subsubsection[{NPOINTS}]{\setlength{\rightskip}{0pt plus 5cm}\#define NPOINTS~50}\label{d2/d80/example_8cc_046c61fd31a06b2051fa0f57e626ee65}




Definition at line 20 of file example.cc.

Referenced by main().\index{example.cc@{example.cc}!SLEEP\_\-LGTH@{SLEEP\_\-LGTH}}
\index{SLEEP\_\-LGTH@{SLEEP\_\-LGTH}!example.cc@{example.cc}}
\subsubsection[{SLEEP\_\-LGTH}]{\setlength{\rightskip}{0pt plus 5cm}\#define SLEEP\_\-LGTH~2}\label{d2/d80/example_8cc_86c9f48acee3e4ad980ffbaacc293a1a}




Definition at line 19 of file example.cc.

\subsection{Function Documentation}
\index{example.cc@{example.cc}!main@{main}}
\index{main@{main}!example.cc@{example.cc}}
\subsubsection[{main}]{\setlength{\rightskip}{0pt plus 5cm}int main (int {\em argc}, \/  char $\ast$ {\em argv}[$\,$])}\label{d2/d80/example_8cc_0ddf1224851353fc92bfbff6f499fa97}




Definition at line 27 of file example.cc.

References Gnuplot::cmd(), NPOINTS, Gnuplot::plot\_\-equation(), Gnuplot::plot\_\-equation3d(), Gnuplot::plot\_\-image(), Gnuplot::plot\_\-slope(), Gnuplot::plot\_\-x(), Gnuplot::plot\_\-xy(), Gnuplot::plot\_\-xy\_\-err(), Gnuplot::plot\_\-xyz(), Gnuplot::replot(), Gnuplot::reset\_\-all(), Gnuplot::reset\_\-plot(), Gnuplot::savetops(), Gnuplot::set\_\-cbrange(), Gnuplot::set\_\-contour(), Gnuplot::set\_\-grid(), Gnuplot::set\_\-hidden3d(), Gnuplot::set\_\-isosamples(), Gnuplot::set\_\-legend(), Gnuplot::set\_\-pointsize(), Gnuplot::set\_\-samples(), Gnuplot::set\_\-smooth(), Gnuplot::set\_\-style(), Gnuplot::set\_\-surface(), Gnuplot::set\_\-title(), Gnuplot::set\_\-xautoscale(), Gnuplot::set\_\-xlabel(), Gnuplot::set\_\-xrange(), Gnuplot::set\_\-ylabel(), Gnuplot::set\_\-yrange(), Gnuplot::set\_\-zlabel(), Gnuplot::set\_\-zrange(), Gnuplot::showonscreen(), Gnuplot::unset\_\-grid(), Gnuplot::unset\_\-legend(), Gnuplot::unset\_\-smooth(), Gnuplot::unset\_\-surface(), Gnuplot::unset\_\-title(), and wait\_\-for\_\-key().

\begin{Code}\begin{verbatim}28 {
29     // if path-variable for gnuplot is not set, do it with:
30     // Gnuplot::set_GNUPlotPath("C:/program files/gnuplot/bin/");
31 
32     // set a special standard terminal for showonscreen (normally not needed),
33     //   e.g. Mac users who want to use x11 instead of aqua terminal:
34     // Gnuplot::set_terminal_std("x11");
35 
36     cout << "*** example of gnuplot control through C++ ***" << endl << endl;
37 
38     //
39     // Using the GnuplotException class
40     //
41     try
42     {
43         Gnuplot g1("lines");
44 
45         //
46         // Slopes
47         //
48         cout << "*** plotting slopes" << endl;
49         g1.set_title("Slopes\\nNew Line");
50 
51         cout << "y = x" << endl;
52         g1.plot_slope(1.0,0.0,"y=x");
53 
54         cout << "y = 2*x" << endl;
55         g1.plot_slope(2.0,0.0,"y=2x");
56 
57         cout << "y = -x" << endl;
58         g1.plot_slope(-1.0,0.0,"y=-x");
59         g1.unset_title();
60 
61         //
62         // Equations
63         //
64         g1.reset_plot();
65         cout << endl << endl << "*** various equations" << endl;
66 
67         cout << "y = sin(x)" << endl;
68         g1.plot_equation("sin(x)","sine");
69 
70         cout << "y = log(x)" << endl;
71         g1.plot_equation("log(x)","logarithm");
72 
73         cout << "y = sin(x) * cos(2*x)" << endl;
74         g1.plot_equation("sin(x)*cos(2*x)","sine product");
75 
76         //
77         // Styles
78         //
79         g1.reset_plot();
80         cout << endl << endl << "*** showing styles" << endl;
81 
82         cout << "sine in points" << endl;
83         g1.set_pointsize(0.8).set_style("points");
84         g1.plot_equation("sin(x)","points");
85 
86         cout << "sine in impulses" << endl;
87         g1.set_style("impulses");
88         g1.plot_equation("sin(x)","impulses");
89 
90         cout << "sine in steps" << endl;
91         g1.set_style("steps");
92         g1.plot_equation("sin(x)","steps");
93 
94         //
95         // Save to ps
96         //
97         g1.reset_all();
98         cout << endl << endl << "*** save to ps " << endl;
99 
100         cout << "y = sin(x) saved to test_output.ps in working directory" << endl;
101         g1.savetops("test_output");
102         g1.set_style("lines").set_samples(300).set_xrange(0,5);
103         g1.plot_equation("sin(12*x)*exp(-x)").plot_equation("exp(-x)");
104 
105         g1.showonscreen(); // window output
106 
107 
108         //
109         // User defined 1d, 2d and 3d point sets
110         //
111         std::vector<double> x, y, y2, dy, z;
112 
113         for (int i = 0; i < NPOINTS; i++)  // fill double arrays x, y, z
114         {
115             x.push_back((double)i);             // x[i] = i
116             y.push_back((double)i * (double)i); // y[i] = i^2
117             z.push_back( x[i]*y[i] );           // z[i] = x[i]*y[i] = i^3
118             dy.push_back((double)i * (double)i / (double) 10); // dy[i] = i^2 / 10
119         }
120         y2.push_back(0.00); y2.push_back(0.78); y2.push_back(0.97); y2.push_back(0.43);
121         y2.push_back(-0.44); y2.push_back(-0.98); y2.push_back(-0.77); y2.push_back(0.02);
122 
123 
124         g1.reset_all();
125         cout << endl << endl << "*** user-defined lists of doubles" << endl;
126         g1.set_style("impulses").plot_x(y,"user-defined doubles");
127 
128         g1.reset_plot();
129         cout << endl << endl << "*** user-defined lists of points (x,y)" << endl;
130         g1.set_grid();
131         g1.set_style("points").plot_xy(x,y,"user-defined points 2d");
132 
133         g1.reset_plot();
134         cout << endl << endl << "*** user-defined lists of points (x,y,z)" << endl;
135         g1.unset_grid();
136         g1.plot_xyz(x,y,z,"user-defined points 3d");
137 
138         g1.reset_plot();
139         cout << endl << endl << "*** user-defined lists of points (x,y,dy)" << endl;
140         g1.plot_xy_err(x,y,dy,"user-defined points 2d with errorbars");
141 
142 
143         //
144         // Multiple output screens
145         //
146         cout << endl << endl;
147         cout << "*** multiple output windows" << endl;
148 
149         g1.reset_plot();
150         g1.set_style("lines");
151         cout << "window 1: sin(x)" << endl;
152         g1.set_grid().set_samples(600).set_xrange(0,300);
153         g1.plot_equation("sin(x)+sin(x*1.1)");
154 
155         g1.set_xautoscale().replot();
156 
157         Gnuplot g2;
158         cout << "window 2: user defined points" << endl;
159         g2.plot_x(y2,"points");
160         g2.set_smooth().plot_x(y2,"cspline");
161         g2.set_smooth("bezier").plot_x(y2,"bezier");
162         g2.unset_smooth();
163 
164         Gnuplot g3("lines");
165         cout << "window 3: log(x)/x" << endl;
166         g3.set_grid();
167         g3.plot_equation("log(x)/x","log(x)/x");
168 
169         Gnuplot g4("lines");
170         cout << "window 4: splot x*x+y*y" << endl;
171         g4.set_zrange(0,100);
172         g4.set_xlabel("x-axis").set_ylabel("y-axis").set_zlabel("z-axis");
173         g4.plot_equation3d("x*x+y*y");
174 
175         Gnuplot g5("lines");
176         cout << "window 5: splot with hidden3d" << endl;
177         g5.set_isosamples(25).set_hidden3d();
178         g5.plot_equation3d("x*y*y");
179 
180         Gnuplot g6("lines");
181         cout << "window 6: splot with contour" << endl;
182         g6.set_isosamples(60).set_contour();
183         g6.unset_surface().plot_equation3d("sin(x)*sin(y)+4");
184 
185         g6.set_surface().replot();
186 
187         Gnuplot g7("lines");
188         cout << "window 7: set_samples" << endl;
189         g7.set_xrange(-30,20).set_samples(40);
190         g7.plot_equation("besj0(x)*0.12e1").plot_equation("(x**besj0(x))-2.5");
191 
192         g7.set_samples(400).replot();
193 
194         Gnuplot g8("filledcurves");
195         cout << "window 8: filledcurves" << endl;
196         g8.set_legend("outside right top").set_xrange(-5,5);
197         g8.plot_equation("x*x").plot_equation("-x*x+4");
198 
199         //
200         // Plot an image
201         //
202         Gnuplot g9;
203         cout << "window 9: plot_image" << endl;
204         const int iWidth  = 255;
205         const int iHeight = 255;
206         g9.set_xrange(0,iWidth).set_yrange(0,iHeight).set_cbrange(0,255);
207         g9.cmd("set palette gray");
208         unsigned char ucPicBuf[iWidth*iHeight];
209         // generate a greyscale image
210         for(int iIndex = 0; iIndex < iHeight*iWidth; iIndex++)
211         {
212             ucPicBuf[iIndex] = iIndex%255;
213         }
214         g9.plot_image(ucPicBuf,iWidth,iHeight,"greyscale");
215 
216         g9.set_pointsize(0.6).unset_legend().plot_slope(0.8,20);
217 
218         //
219         // manual control
220         //
221         Gnuplot g10;
222         cout << "window 10: manual control" << endl;
223         g10.cmd("set samples 400").cmd("plot abs(x)/2"); // either with cmd()
224         g10 << "replot sqrt(x)" << "replot sqrt(-x)";    // or with <<
225 
226         wait_for_key();
227 
228     }
229     catch (GnuplotException ge)
230     {
231         cout << ge.what() << endl;
232     }
233 
234 
235     cout << endl << "*** end of gnuplot example" << endl;
236 
237     return 0;
238 }
\end{verbatim}
\end{Code}


\index{example.cc@{example.cc}!wait\_\-for\_\-key@{wait\_\-for\_\-key}}
\index{wait\_\-for\_\-key@{wait\_\-for\_\-key}!example.cc@{example.cc}}
\subsubsection[{wait\_\-for\_\-key}]{\setlength{\rightskip}{0pt plus 5cm}void wait\_\-for\_\-key ()}\label{d2/d80/example_8cc_621288a08ecc9633729935737256e831}




Definition at line 242 of file example.cc.

Referenced by main().

\begin{Code}\begin{verbatim}243 {
244 #if defined(WIN32) || defined(_WIN32) || defined(__WIN32__) || defined(__TOS_WIN__)  // every keypress registered, also arrow keys
245     cout << endl << "Press any key to continue..." << endl;
246 
247     FlushConsoleInputBuffer(GetStdHandle(STD_INPUT_HANDLE));
248     _getch();
249 #elif defined(unix) || defined(__unix) || defined(__unix__) || defined(__APPLE__)
250     cout << endl << "Press ENTER to continue..." << endl;
251 
252     std::cin.clear();
253     std::cin.ignore(std::cin.rdbuf()->in_avail());
254     std::cin.get();
255 #endif
256     return;
257 }
\end{verbatim}
\end{Code}


